\documentclass[10pt,twocolumn]{article}

% use the oxycomps style file
\usepackage{oxycomps}
\usepackage{xcolor}

\pdfinfo{
    /Title (An Introductory Tutorial to Git)
    /Author (Oliver Wilkins)
}

% set the title and author information
\title{An Introductory Tutorial to Git}
\author{Oliver Wilkins}
\affiliation{Occidental College}
\email{wilkins@oxy.edu}

\begin{document}

\maketitle

\section{Introduction}

This tutorial will provide a basic introduction to git, with the intended audience being someone who is new to version control and wishes to use it for a long-term project. It assumes you will be using a command line interface on mac, with GitHub as the location for the remote repository. If you do not have a GitHub account, create one prior to following the tutorial. 

Git is a version control system which helps developers keep their work organized by keeping track of what has been changed, when, and by whom.  It works by storing files in four locations. One is the remote repository (on GitHub), and three are local: the working directory, the staging area, and the local repository. In this tutorial you will learn how to move files between these locations, as well as how to manage different versions of your work using stashing, branching, and merging.


\section{Initialize local Repository}

In this tutorial we will be initializing the repository locally using git init and connecting it to remote later, but it is also possible to initialize the repository on github and use git clone to make a local repository. 

In terminal, create a new directory for your project and navigate to it. Add a file file1.txt and write the letters a, b, and c, each on a separate line.\\\\
\textcolor{cyan}{
mkdir git-tutorial\\
cd git-tutorial\\
touch file1.txt\\
}

In the directory, initialize a local repository.\\\\
\textcolor{cyan}{
git init\\
}

The git-tutorial directory has now been transformed into a git repository. This means we can move its files to the staging area and local repository, before ultimately pushing them to remote. 


\subsection{First Commit}

Let’s move our text file to the staging area. Any file we want to put in our local repository must go through the staging area first. \\\\
\textcolor{cyan}{
git add file1.txt\\
}

Now that our text file is in the staging area we can add it to our local repository via a commit.\\\\
\textcolor{cyan}{
git commit\\
}

You will be prompted to add a comment to your commit. Write something about what you are changing, like “added file1, with a,b,c.” 

\subsection{Connecting and pushing to Remote}

Now that we have a local repository with a commit, it is time to connect to remote. In GitHub, create an empty repository named git-tutorial. 

Back in terminal, type the following command:\\\\
\textcolor{cyan}{
git remote add origin https://github.com/yourUsername/git-tutorial.git\\
}

This command adds the GitHub repository we just created as a remote associated with the local repository. ‘origin’ is a name we are assigning to the url so that we do not have to type it out again. We will now ‘push’ our local repository to remote. \\\\
\textcolor{cyan}{
git push --set-upstream origin main\\
}

In future pushes you will only need to type ‘git push’, the rest of the command is doing setup work that only needs to be done once. 

Go back to GitHub and refresh the page. You should now see your file.txt in the repository. 


\subsection{Pulling from Remote}

If you are working with other people, they may make changes to the remote repository which you want to see reflected in your local repository. To mimic this, we will add our own file in GitHub and ‘pull’ it to local. In your github repository, click on ‘add file’. Name it new-file.txt and write whatever you want in it. Click ‘commit changes’.

Back in terminal, pull from remote.\\\\
\textcolor{cyan}{
git pull\\
}

New-file.txt is now in your local repository.


\subsection{Stashing}

Stashing is a technique which sets aside something you are working on so that you can have a clean working directory but still come back later to the thing you were working on. 

Create a file stash-file.txt and add it to the staging area.\\\\
\textcolor{cyan}{
touch stash-file.txt\\
git add stash-file.txt\\
}

Now let’s stash this file away to come back to later. \\\\
\textcolor{cyan}{
git stash \\
}

If you run the ls command, you will see that stash-file.txt no longer shows up in the working directory. To see the stashes we have made, use the stash show command. \\\\
\textcolor{cyan}{
git stash show\\
}

You will see the stash we just made. To add the stashed file back to our working directory, use stash pop.\\\\
\textcolor{cyan}{
git stash pop\\
}

The stash-file.txt file is now back in our working directory (and staging area). Since we do not want this file to be committed, let’s move it out of the staging area. We will use git status to track the contents of the staging area before and after to ensure it works.\\\\
\textcolor{cyan}{
git status\\
git reset\\
git status\\
}

The stash file is no longer in the staging area, but is still in the local directory. 


\subsection{Branching}

Branching can be used to make changes in a way that does not impact the main version of your work. We will make a new branch called new-branch.\\\\
\textcolor{cyan}{
git branch new-branch\\
git branch\\
}

The git branch command (with no additional arguments) will show a list of all branches, and indicate which one you are currently on. It should show you as currently being on the main branch. Let’s switch to the new one. \\\\
\textcolor{cyan}{
git switch new-branch\\
}

In the new branch, edit the file1.txt file by adding an e on the line below c. Add and commit the file, then switch back to the main branch.\\\\
\textcolor{cyan}{
git add file1.txt\\
git commit\\
git switch main\\
}

Back in the main branch, edit file1.txt again, this time adding a d on the line below c. Add and commit.\\\\
\textcolor{cyan}{
git add file1.txt\\
git commit\\
}

We now have two branches with two different versions of file1.txt.


\subsection{Merging }

Let’s merge our new branch into our main branch so that the changes we have made can be consolidated into one place. Make sure you are still in the main branch.\\\\
\textcolor{cyan}{
git merge new-branch\\
}

This command should generate an error. This is because both branches made changes to the same file in a way that git does not know how to handle. To fix this, open file1.txt. You will see both d and e in the file, with messages indicating which branch each came from. Edit the file so the letters are in alphabetical order. Add and commit the edited file, then push to remote. \\\\
\textcolor{cyan}{
git add file1.txt\\
git commit\\
git push\\
}

The GitHub repository will now reflect the combined work of both branches. 


\end{document}
